\documentclass{article}
% Set target color model to RGB
\usepackage[inner=2.0cm,outer=2.0cm,top=2.5cm,bottom=2.5cm]{geometry}
\usepackage{setspace}
\usepackage[rgb]{xcolor}
\usepackage{verbatim}
\usepackage{amsmath}
\usepackage{subcaption}
\usepackage{amsgen,amsmath,amstext,amsbsy,amsopn,tikz,amssymb,tkz-linknodes}
\usepackage{fancyhdr}
\usepackage[colorlinks=true, urlcolor=blue,  linkcolor=blue, citecolor=blue]{hyperref}
\usepackage[colorinlistoftodos]{todonotes}
\usepackage{rotating}
\usepackage{listings}
\usepackage{amsmath,amsfonts,amssymb}
\usepackage{amsmath}
\lstset{
%	language=bash,
	basicstyle=\ttfamily
}

\newcommand{\ra}[1]{\renewcommand{\arraystretch}{#1}}

\newtheorem{thm}{Theorem}[section]
\newtheorem{prop}[thm]{Proposition}
\newtheorem{lem}[thm]{Lemma}
\newtheorem{cor}[thm]{Corollary}
\newtheorem{defn}[thm]{Definition}
\newtheorem{rem}[thm]{Remark}
\numberwithin{equation}{section}
\graphicspath{ {./img/} }

\newcommand{\homework}[6]{
   \pagestyle{myheadings}
   \thispagestyle{plain}
   \newpage
   \setcounter{page}{1}
   \noindent
   \begin{center}
   \framebox{
      \vbox{\vspace{2mm}
    \hbox to 6.28in { {\bf F1TENTH Autonomous Racing \hfill {\small (#2)}} }
       \vspace{6mm}
       \hbox to 6.28in { {\Large \hfill #1  \hfill} }
       \vspace{6mm}
       \hbox to 6.28in { {\it Instructor: {\rm #3} \hfill Name: {\rm #5}, StudentID: {\rm #6}} }
       %\hbox to 6.28in { {\it T\textbf{A:} #4  \hfill #6}}
      \vspace{2mm}}
   }
   \end{center}
   \markboth{#5 -- #1}{#5 -- #1}
   \vspace*{4mm}
}


\newcommand{\problem}[3]{~\\\fbox{\textbf{Problem #1: #2}}\hfill (#3 points)\newline}
\newcommand{\subproblem}[1]{~\newline\textbf{(#1)}}
\newcommand{\D}{\mathcal{D}}
\newcommand{\Hy}{\mathcal{H}}
\newcommand{\VS}{\textrm{VS}}
\newcommand{\solution}{~\newline\textbf{\textit{(Solution)}} }

\newcommand{\bbF}{\mathbb{F}}
\newcommand{\bbX}{\mathbb{X}}
\newcommand{\bI}{\mathbf{I}}
\newcommand{\bX}{\mathbf{X}}
\newcommand{\bY}{\mathbf{Y}}
\newcommand{\bepsilon}{\boldsymbol{\epsilon}}
\newcommand{\balpha}{\boldsymbol{\alpha}}
\newcommand{\bbeta}{\boldsymbol{\beta}}
\newcommand{\0}{\mathbf{0}}


\usepackage{booktabs}



\begin{document}

	\homework {Lab 5: Scan Matching}{Due Date:}{INSTRUCTOR}{}{STUDENT NAME}{ID}
	\thispagestyle{empty}
	% -------- DO NOT REMOVE THIS LICENSE PARAGRAPH	----------------%
	\begin{table}[h]
		\begin{tabular}{l p{14cm}}
		\raisebox{-2cm}{} & \textit{This lab and all related course material on \href{http://f1tenth.org/}{F1TENTH Autonomous Racing} has been developed by the Safe Autonomous Systems Lab at the University of Pennsylvania (Dr. Rahul Mangharam). It is licensed under a \href{https://creativecommons.org/licenses/by-nc-sa/4.0/}{Creative Commons Attribution-NonCommercial-ShareAlike 4.0 International License.} You may download, use, and modify the material, but must give attribution appropriately. Best practices can be found \href{https://wiki.creativecommons.org/wiki/best_practices_for_attribution}{here}.}
		\end{tabular}
	\end{table}
	% -------- DO NOT REMOVE THIS LICENSE PARAGRAPH	----------------%
	
	\noindent \large{\textbf{Course Policy:}} Read all the instructions below carefully before you start working on the assignment, and before you make a submission. All sources of material must be cited. The University Academic Code of Conduct will be strictly enforced.
	\\
	\\
	\textbf{THIS IS A GROUP ASSIGNMENT}. Submit one from each team.\\

	 \section{Theoretical Questions}
	 \begin{enumerate}
	 	\item $M_{i}=\left(\begin{array}{cccc}{1} & {0} & {p_{i 0}} & {-p_{i 1}} \\ {0} & {1} & {p_{i 1}} & {p_{i 0}}\end{array}\right)$
 		\begin{enumerate}
 		\item Show that $B_{i} :=M_{i}^{T} M_{i}$ is symmetric. \\
 		\item Demonstrate that $B_{i}$ is positive semi-definite. \\
 		\newline{}
         \textbf{Solution:}
         \newline{}
         \newline{}
         $M_i$ = 
         $\begin{bmatrix} 
            1 & 0 & p_{i0} & -p_{i1}\\
            0 & 1 & p_{i1} & p_{i0}
        \end{bmatrix}$
        \newline{}
        $B_i$ = 
        $\begin{bmatrix} 
            1 & 0 & p_{i0} & -p_{i1}\\
            0 & 1 & p_{i1} & p_{i0}\\
            p_{i0} & p_{i1} & p_{i0}^2 + p_{i1}^2 & 0\\
            -p_{i1} & p_{i0} & 0 & p_{i0}^2 + p_{i1}^2
        \end{bmatrix}$
        \newline{}
        \newline{}
        The eigenvalues of the matrix $B_i$ are $0$, $0$, $p_{i0}^2 + p_{i1}^2 + 1$ and $p_{i0}^2 + p_{i1}^2 + 1$. All these values are for sure greater or equal to zero, therefore the matrix $B_i$ is positive semi-definite and obviously symmetric as proven by construction.	
 		
	 	\end{enumerate}
 		
 		\item The following is the optimization problem:
 		\[ 
 		x^{*}=\operatorname{argmin}_{x \in \mathbb{R}^{4}} \sum_{i=1}^{n}\left\|M_{i} x-\pi_{i}\right\|_{2}^{2} \quad \text { s.t. } \quad x_{3}^{2}+x_{4}^{2}=1
 		\] 
 		\begin{enumerate}
		\item Find the matrices M, W and g which give you the formulation 
		\[
		 x^{*}=\operatorname{argmin}_{x \in \mathbb{R}^{4}} x^{T} M x+g^{T} x 
		\quad \text { s.t. } x^{T} W x=1
		\]
 		\textbf{Solution:}
		\\
		\\
		Answer here
		\\
		\\		
		
		\item Show that M and W are positive semi definite.
 		\textbf{Solution:}
		\\
		\\
		Answer here
		\\
		\\		
		
 		\end{enumerate}
	 	
	 \end{enumerate}


			
\end{document} 